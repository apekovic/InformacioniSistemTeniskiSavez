\documentclass{article}
\usepackage[utf8]{inputenc}
\usepackage[T1]{fontenc}
\usepackage{longtable}

\begin{document}
        \begin{longtable}{| p{.20\textwidth} | p{.80\textwidth} |} 
            \hline
                Kratak opis & Korisnik vrši registraciju kao igrač tako što treba da zakaže sportski/medicinski pregled, unese šifru rezultata pregleda koji dobije i uplati članarinu.\\ 
            \hline    
                Učesnici & \begin{itemize}
                    \item Korisnik koji se registruje.
                    \item Administrator koji kreira nalog korisinka.
                    \item Teniski Savez koji verifikuje podatke.
                    \item Medicinska ustanova u kojoj se zakazuje pregled i koja šalje rezultate.
                \end{itemize} \\
            \hline
               Preduslovi & \begin{enumerate}
                   \item Sistem je u funkciji.
                   \item Korisnik nije registrovan.
               \end{enumerate}\\
            \hline  
                Postuslovi & \begin{enumerate}
                    \item Korisnik je uspešno postao igrač, tj. član Teniskog Saveza.
                    \item Baza podataka je ažurirana unosom novog igrača u tabelu svih igrača, kao i unosom igrača u tabelu aktivnih igrača u tekućoj godini.
                \end{enumerate}\\
            \hline
                Osnovni tok & \begin{enumerate}
                    \item Korisnik otvara formu za registraciju
                    \item Sistem prikazuje stranicu sa svim podacima bitnim za registraciju u okviru Teniskog Saveza, kao i opciju da se registruje.
                    \item Korisnik bira da se registruje.
                    \item Sistem izbacuje formu sa opcijama tipa člana za registraciju.
                    \item Korisnik bira tip "Igrač".
                    \item Sistem izbacuje formu sa pitanjem "Da li je obavljen medicinski pregled?" i dugmiće sa odgovorima "Da" i "Ne".
                    \item Ako korisnik izabere odgovor "Ne":
                    \begin{enumerate}
                        \item[5.1] Korisnik dobija od sistema formu sa pitanjem "Da li želite da zakažete pregled?".
                        \item[5.2] Korisnik odgovara sa "Da".
                        \item[5.3] Prelazi se na slučaj upotrebe 3.1.3 Zakazivanje medicinskog pregleda.
                        \item[5.4] Nakon uspešnog zakazivanja pregleda prelazi se na slučaj upotrebe 3.7.1 Unosi rezultat pregleda igrača.
                    \end{enumerate}
                    \item Ako korisnik izabere odgovor "Da":
                    \begin{enumerate}
                        \item[6.1] Sistem traži unos šifre pregleda.
                        \item[6.2] Korisnik unosi šifru pregleda.
                        \item[6.3] Sistem prikazuje nov prozor sa poljima za popunjavanje, čekiranje ili izmenu, kao 5 opcija za kačenje fajlova:
                        \begin{itemize}
                            \item "Potvrda o uplati" da se doda prilog koji sadrži potvrdu o uplati za izradu legitimacije.
                            \item "Potvrda o članstvu u klubu" da se doda dokument sa podacima o igraču i klubu za koji igra, koji je pečatiran.
                            \item "Slika" da se doda mala slika koja će biti na takmičarskoj legitimaciji.
                            \item "Kopija izvoda iz matične knjige".
                            \item "Dokument o obavljenom pregledu".
                        \end{itemize}  
                        \item[6.4] Korisink popunjava polja, opciono menja popunjena i dodaje priloge.
                        \item[6.5] Korisnik potvr\dj uje izmene.
                    \end{enumerate}
                 \end{enumerate}\\
            \hline
                Osnovni tok &
                \begin{enumerate}
                    \item[8.] (Nastavak)  Ako korisnik izabere odgovor "Da":
                    \begin{enumerate}
                        \item[6.6] Sistem šalje Teniskom Savezu obaveštenje o prispelom zahtevu za članstvo radi verifikacije podataka.
                        \item[6.7] Teniski Savez potvr\dj uje validnost podataka.
                        \item[6.8] Sistem šalje administratoru obaveštenje o uspešno ispoštovanim obaveznim početnim koracima za registraciju.
                        \item[6.9] Administrator šalje korisniku e-mail sa:
                        \begin{itemize}
                            \item Detaljima prispeća kartice nakon njene izrade.
                            \item Linkom (vremenski ogranicenim) privremenog naloga koji mora da se popuni podacima za registraciju (mail i šifra) i potvrdi.
                        \end{itemize}
                        \item[6.10] Igrač popunjava podatke i potvr\dj uje popunjavanje forme.
                        \item[6.11] Administrator dodeljuje igraču njegov registracioni broj koji će biti broj legitimacije.
                        \item[6.12] Administrator evidentira nalog kao aktivan.
                    \end{enumerate}
                \end{enumerate}\\
            \hline
                Alternativni tokovi & 
                \begin{itemize}
                    \item[A3.] Ako korisnik izabere tip "Klub" svodi se na slučaj upotrebe 3.1.2 Registruje se kao klub. 
                    \item[A5.2] Igrač je odgovorio sa "Ne" i zakazivanje se završava sa obaveštenjem "Registracija Igrača se može završiti tek nakon obavljenog pregleda.".
                    \item[A6.2] \begin{itemize}
                        \item Ako korisnik ne unese validnu šifru registracija se završava neuspešno.
                        \item Ako je za datu šifru u bazi rezultat pregleda "Pao", korisniku se šalje mail da registracija nije moguća.
                        \item Ako nije okačio dokument o obavljenom pregledu, sistem izbacuje obaveštenje da ne može da pre\dj e na sledeći korak bez tog dokumenta.
                    \end{itemize} 
                    \item[A6.3] Ako igrač pritisne na dugme "Odustani" registracija se neuspešno završava.
                    \item[A6.5] Ako se ne popune obavezna polja i ne dodaju sva dokumenta, sistem izbacuje obaveštenje da polja moraju biti popunjena, kao i da moraju sva dokumenta da se prilože. Sistem nas vraća nazad na korak 6.3. 
                    \item[A6.7] Teniski Savez može konstatovati lažnu ili neispravnu uplatu i poslati korisniku e-mail da registracija može da se izvrši tek nakon izvršene uplate.
                    \item[A6.9] Ako link istekne, korisnik može odgovoriti na isti mail sa zahtevom novog linka i proces se nastavlja ponavljanjem koraka 6.9.
                \end{itemize}\\
            \hline
                Podtokovi & /\\
            \hline
                Specijalni zahtevi & Korisnik mora da ima e-mail adresu.\\
            \hline 
                Dodatne informacije &  \begin{itemize} 
                \item U koraku 6.3 polja su: 
                    \begin{itemize}
                        \item Ime (Popunjeno)
                        \item Prezime (Popunjeno)
                        \item JMBG
                        \item Datum ro\dj enja
                        \item Kvadratić "Dvojno državljanstvo" da obeleži ako je odgovor "Da".
                        \item E-mail (Popunjeno)
                        \item Telefon
                        \item Grad (Padajući meni)
                        \item Region (Padajući meni)
                        \item Klub (Padajući meni na osnovu izabranog Grada ili Regiona)
                    \end{itemize}
                \item Popunjena polja vuku podatke iz baze. To su podaci koje je medicinska ustanova unela kroz formu.
                \item Neobavezna polja su "Dvojno državljanstvo" i jedno od "Grad" i "Region" (Makar jedno mora biti izabrano).  Ako se izabere Grad, Region se neće gledati u filtriranju klubova.
                \item Visina uplate za izradu legitimacije je definisana pravilnikom.
            \end{itemize} \\
        \hline
        \caption{Korisnik se registruje kao igrač} 
        \end{longtable}

\end{document}