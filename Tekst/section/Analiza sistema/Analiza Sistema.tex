\documentclass{article}
\usepackage[utf8]{inputenc}
\usepackage[T1]{fontenc}

\begin{document}
\section{Analiza sistema}
\subsection{Kratak uvod}
Ovaj rad predstavlja predlog Informacionog sistema za Teniski Savez Srbije koji teži da pomogne realizaciji automatizacije procesa u okviru Saveza, koji i dalje većinu svojih procesa obavlja na teži način. Rad je ra\dj en kao projekat iz predmeta "Informacioni sistemi" na Matematičkom fakultetu Univerziteta u Beogradu. Domen kojim se bavi je takmičarski deo, tj. organizacija domaćih turnira na teritoriji Srbije i to samo od strane klubova u okviru Teniskog Saveza, a tipovi turnira bi bili pojedinačni. 
Preciznije, informacioni sistem pokriva sledeće nadležnosti Takmičarskog odbora:
\begin{itemize}
\item Stara se o organizaciji teniskih takimičenja.
\item Usvaja kalendar takmičenja.
\item Brine se o izdavanju rang liste i verifikuje godišnju rang listu.
\item Stara se o sprovođenju postupka za registraciju tenisera i klubova.
\item Stara se o svim ostalim pitanjima vezanim za takmičenja na teritoriji RS.   
\end{itemize}


\subsection{Korisnici sistema}
\begin{enumerate}
        \item \textbf{Korisnik} - Osoba (ili osoblje kluba) koja aplicira za članstvo u Savezu (registrovani korisnik) ili koja samo želi da prati javno dostupne informacije o turnirima i igračima (neregistrovani korisnik).
        \item \textbf{Igrač} - Registrovani korisnik (član Saveza) koji želi da učestvuje na turnirima.
        \item \textbf{Klub} - Registrovani korisnik (član Saveza) koji može imati ulogu organizatora turnira i koji ima svoje igrače koji se prijavljuju na turnire. 
        \item \textbf{Organizator} - Klub koji je dobio od Saveza dozvolu da organizuje turnir.
        \item \textbf{Teniski Savez} - Objedinjeni naziv za osoblje izabrano od strane Teniskog Saveza za prijem uživo korisinka (neregistrovanih), igrača i klubova (registrovanih) za sve procese koje žele da obave u okviru Saveza, kao i za korespondenciju preko sistema.
        \item \textbf{Administrator} - Osoba iz Teniskog Saveza koja vrši davanje dozvola ili ih ukida, kreira naloge ili proširuje funkcionalnosti naloga korisnika.
        \item \textbf{Medicinska ustanova} - Osoblje Medicinske ustanove koja je licencirana od strane Teniskog Saveza za preglede igrača. Umrežena je sa sistemom radi lakšeg zakazivanja pregleda i bržeg dobijanja rezultata.

    \end{enumerate}
    
\subsection{Kratak opis analize sistema}
Proučavanjem Teniskog Saveza Srbije je ustanovljeno da je njihov trenutni sistem zastareo i nedovoljno automatizovan. I dalje se većina dokumenata čuva na papiru i skenira, pa tako kači na veb stranicu. Igrači se registruju kao članovi Saveza tako što idu i sami se prijavljuju uživo ili neko umesto njih ide u Savez da ih prijavi. Tako\dj e , mogu se prijaviti i preko kluba, ali klub tada radi pešački posao. Ovaj način rada prouzrokuje nepotrebno trošenje vremena svih aktera u okviru Saveza. Baza podataka deluje funkcionalno samo za pregled rang listi, što bi trebalo proširiti na sve aspekte čuvanja podataka o igračima, klubovima i rezultatima kroz godine, kao i svom pomoćnom osoblju koje učestvuje u fazama organizacije turnira.\par
Ovaj informacioni sistem omogućava lakše prijavljivanje novih članova u okviru Saveza, lakše organizovanje medicinskih pregleda, lakšu prijavu za organizaciju turnira i adekvatniji izbor kluba u skladu sa tipom turnira koji treba da se organizuje, lakšu prijavu na turnire i uvid u njihove detalje organizovanja. Tako\dj e , veći uvid igrača u njihove rezultate radi mogućeg poboljšanja taktike pripreme za turnir ili neki pojedini meč. Savez bi u svakom trenutku imao uvid u sve pojedinosti organizovanja svakog turnira i ko je sve učestvovao u svakom procesu realizacije. Sve izmene podataka bi bile lakše i manje papirologije bi bilo u opticaju. Jasno bi se znala uloga svakog člana, efikasnije bi se vršile dodele pomoćnog osoblja, jer bi se znalo ko je najadekvatniji da se dodeli organizatoru. \par
U okviru sistema se ne vodi evidencija ni o čemu što ne interesuje direktno Teniski Savez, tj Takmičarski odbor. 

\end{document}
