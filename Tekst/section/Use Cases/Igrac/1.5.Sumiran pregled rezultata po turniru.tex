\documentclass{article}
\usepackage{longtable}
\begin{document}
%\begin{center}
%\begin{tabularx}{\textwidth}{|1|X|}
\begin{longtable}{| p{.20\textwidth} | p{.80\textwidth} |} 
\hline
    Kratak opis & Igrac podnosi zahtev za pregled svih meceva po turniru.\\ 
\hline    
    Učesnici & Registrovani igraci
    \\
\hline
   Preduslovi & \begin{enumerate}
       \item Sistem je u funkciji.
       \item Igrac je registrovan
   \end{enumerate}\\
\hline  
    Postuslovi & \begin{enumerate}
        \item Igrac je dobio pregled svih meceva po turniru
    \end{enumerate}\\
\hline
    Osnovni tok & \begin{enumerate}
        \item Igrac otvara stranicu TSS-a
        \item Igrac bira karticu 'Turniri', zatim 'Dosadasnji rezultati'
        \item Igrac filtrira pretragu po turnirima
        \item Sistem prikazuje osvojene bodove na turniru, protivnika, rezultate po setovima i statistiku meca za svako kolo, tj. svaki mec na turniru
    \end{enumerate}\\
\hline
    Alternativni tokovi & \begin{itemize}
        \item[A3] Igrac nije odigrao nijedan mec: Sistem obavestava korisnika da nije odigrao nijedan mec. Proces se zavrsava
    \end{itemize}\\
\hline
    Podtokovi & /\\
\hline
    /\\
\hline
    Dodatne informacije & \begin{itemize}
        \item Bodovi se dobijaju nakon prvog kola
        \item Informacije koje korisnik dobija o protivniku su: Ime, Prezime, Klub za koji igra i mesto koje zauzima na rang listi
        \item Novcana nagrada moze biti i 0, ukoliko turnir ne obezbedjuje novcanu nagradu
        \item Statistika meca obuhvata broj odigranih setova, broj dobijenih i izgubljenih setova, broj dobijenih i izgubljenih gemova.
    \end{itemize}\\
\hline
%\end{tabularx}
%\end{center}    
\caption{Pregled rezultata po turniru} % needs to go inside longtable environment
%\label{tab:myfirstlongtable}
\end{longtable}


%\begin{figure}[!ht]
%\begin{center}
%\includegraphics[scale=0.55]{sections/images/Dijagram_aktivnsti_registracije_uzivo.jpg}
%\end{center}
%\caption{Dijagram pregleda rezultata po turniru}
%\label{fig:kontekst}
%\end{figure}
\end{document}