\documentclass{article}
\usepackage[utf8]{inputenc}
\usepackage[T1]{fontenc}
\usepackage{longtable}

\begin{document}
        \begin{longtable}{| p{.20\textwidth} | p{.80\textwidth} |} 
            \hline
                Kratak opis & Korisnik vrši registraciju kao igrač tako što treba da zakaže sportski/medicinski pregled, pošalje sva neophodna dokumenta pečatirana, sliku, dokaz o uplati članarine i ostavi lične podatke.\\ 
            \hline    
                Učesnici & \begin{itemize}
                    \item Korisnik koji se registruje.
                    \item Teniski Savez koji verifikuje podatke.
                \end{itemize} \\
            \hline
               Preduslovi & \begin{enumerate}
                   \item Sistem je u funkciji.
                   \item Korisnik nije registrovan.
               \end{enumerate}\\
            \hline  
                Postuslovi & \begin{enumerate}
                    \item Korisnik je uspešno postao igrač, tj. član Teniskog Saveza.
                    \item Baza podataka je ažurirana unosom novog igrača u tabelu svih igrača, kao i unosom igrača u tabelu aktivnih igrača u tekućoj godini.
                \end{enumerate}\\
            \hline
                Osnovni tok & \begin{enumerate}
                    \item Korisnik otvara formu za registraciju.
                    \item Sistem prikazuje stranicu sa svim podacima bitnim za registraciju u okviru Teniskog Saveza, kao i opciju da se registruje.
                    \item Korisnik bira da se registruje.
                    \item Sistem izbacuje formu sa opcijama tipa člana za registraciju.
                    \item Korisnik bira tip "Igrač".
                    \item Sistem izbacuje formu sa pitanjem "Da li je obavljen medicinski pregled?" i dugmiće sa odgovorima "Da" i "Ne".
                    \item Ako korisnik izabere odgovor "Ne":
                    \begin{enumerate}
                        \item[5.1] Korisnik dobija od sistema formu sa pitanjem "Da li želite da zakažete pregled?".
                        \item[5.2] Korisnik odgovara sa "Da".
                        \item[5.3] Prelazi se na slučaj upotrebe 3.1.3 Zakazuje medicinski pregled.
                        \item[5.4] Korisnik se vraća na korak 1 nakon dobijenih rezultata.
                    \end{enumerate}
                    \item Ako korisnik izabere odgovor "Da":
                    \begin{enumerate}
                        \item[6.1] Sistem prikazuje nov prozor sa poljima za popunjavanje, čekiranje, kao 5 opcija za kačenje fajlova:
                        \begin{itemize}
                            \item "Potvrda o uplati" da se doda prilog koji sadrži potvrdu o uplati za izradu legitimacije.
                            \item "Potvrda o članstvu u klubu" da se doda dokument sa podacima o igraču i klubu za koji igra, koji je pečatiran.
                            \item "Slika" da se doda mala slika koja će biti na takmičarskoj legitimaciji.
                            \item "Kopija izvoda iz matične knjige".
                            \item "Dokument o obavljenom pregledu".
                        \end{itemize}  
                    \end{enumerate}
                 \end{enumerate}\\
            \hline
                Osnovni tok &
                \begin{enumerate}
                    \item[8.] (Nastavak)  Ako korisnik izabere odgovor "Da":
                    \begin{enumerate}
                        \item[6.2] Korisink popunjava polja i dodaje priloge.
                        \item[6.3] Korisnik potvr\dj uje izmene.
                        \item[6.4] Sistem šalje Teniskom Savezu obaveštenje o prispelom zahtevu za članstvo radi verifikacije podataka.
                        \item[6.5] Teniski Savez potvr\dj uje validnost podataka.
                        \item[6.6] Sistem šalje korisniku e-mail sa linkom (vremenski ograničenim) privremenog naloga koji mora da se popuni podacima za registraciju (mail i šifra) i potvrdi. Takođe dugme "Pošalji ponovo" u slučaju da link istekne.
                        \item[6.7] Igrač popunjava podatke i potvr\dj uje popunjavanje forme.
                        \item[6.8] Sistem evidentira nalog kao aktivan.
                    \end{enumerate}
                \end{enumerate}\\
            \hline
                Alternativni tokovi & 
                \begin{itemize}
                    \item[A3] Ako korisnik izabere tip "Klub" svodi se na slučaj upotrebe 3.1.2 Registruje se kao klub. 
                    \item[A5.2] Korisnik je odgovorio sa "Ne" i zakazivanje se završava sa obaveštenjem "Registracija Igrača se može završiti tek nakon obavljenog pregleda.".
                    \item[A6.1] Ako korisnik pritisne na dugme "Odustani" registracija se neuspešno završava.
                    \item[A6.3] Ako se ne popune obavezna polja i ne dodaju sva dokumenta, sistem izbacuje obaveštenje da polja moraju biti popunjena, kao i da moraju sva dokumenta da se prilože. Sistem nas vraća nazad na korak 6.1. 
                    \item[A6.5] Teniski Savez može konstatovati neispravne ili nepotpune podatke. Tada sistem šalje korisniku e-mail da registracija nije uspešna.
                    \item[A6.8] Ako link istekne, korisnik može da pritisne dugme "Pošalji ponovo" čime se vraća na korak 6.6. Ako link istekne, a korisnik ne zatraži novi i istekne 6 meseci, registracija je neuspešna.
                \end{itemize}\\
            \hline
                Podtokovi & /\\
            \hline
                Specijalni zahtevi & Korisnik mora da ima e-mail adresu.\\
            \hline 
                Dodatne informacije &  \begin{itemize} 
                \item U koraku 6.3 polja su: 
                    \begin{itemize}
                        \item Ime
                        \item Prezime
                        \item JMBG
                        \item Datum ro\dj enja
                        \item Kvadratić "Dvojno državljanstvo" da obeleži ako je odgovor "Da".
                        \item E-mail
                        \item Telefon
                        \item Adresa
                        \item Grad 
                        \item Region (Padajući meni)
                        \item Klub (Padajući meni na osnovu izabranog Grada ili Regiona)
                    \end{itemize}
                \item Neobavezna polja su "Dvojno državljanstvo" i jedno od "Grad" i "Region" (Makar jedno mora biti izabrano).  Ako se izabere Grad, Region se neće gledati u filtriranju klubova.
            \end{itemize} \\
        \hline
        \caption{Korisnik se registruje kao igrač} 
        \end{longtable}

\end{document}