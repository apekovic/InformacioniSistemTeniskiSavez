\documentclass{article}
\usepackage[utf8]{inputenc}
\setlength{\parindent}{10ex}
\usepackage{indentfirst}

% \title{}


\begin{document}


% \maketitle
\section{Uvod}
Ovaj rad predstavlja predlog Informacionog sistema za Teniski Savez Srbije koji tezi da pomogne realizaciji automatizacije procesa u okviru Saveza, koji i dalje vecinu svojih procesa obavlja na tezi nacin. Rad je radjen kao projekat iz predmeta "Informacioni sistemi" na Matematickom fakultetu Univerziteta u Beogradu. Domen kojim se bavi je takmicarski deo, tj. organizacija domacih turnira na teritoriji Srbije i to samo od strane klubova u okviru Teniskog Saveza i tipovi turnira bi bili pojedinacni. 
Najpreciznije, informacioni sistem pokriva sledece nadleznosti Takmicarskog odbora
\begin{itemize}
\item Stara se o organizaciji teniskih takimicenja
\item Usvaja kalendar takmicenja
\item Brine se o izdavanju rang liste i verifikuje godisnju rang listu
\item Stara se o sprovodjenju postupka za registraciju tenisera i klubova
\item Stara se o svim ostalim pitanjima vezanim za takmicenja na teritoriji RS   
\end{itemize}


\subsection{Korsnici sistema}
\begin{enumerate}
        \item Neregistrovani Korisnik - Osoba koja zeli da postane clan Saveza ili samo zeli da prati dogadjanje na Turnirima i rangiranje Igraca u okviru Saveza.
        \item Igrac - Clan Saveza koji zeli da ucestvuje na turnirima.
        \item Klub - Clan Saveza koji moze imati ulogu organizatora turnira i koji ima svoje igrace koji se prijavljuju na turnire. 
        \item Administrator - Osoba iz Teniskog Saveza Srbije, koja u dogovoru sa Takmicarskim Odborom vrsi davanje dozvola ili ih uklanja, kreira naloge ili prosiruje funkcionalnosti naloga korisnika.
        \item Tehnicko osoblje - Osoblje koje je Savez dodelio Klubu, koji organizuje turnir, radi upisivanja evidencija o turniru u sistem. 
        \item Medicinsko osoblje - Osoblje koje radi u okviru Medicinskih ustanova koje su umrezene sa Teniskim Savezom radi lakse organizacije pregleda igraca i oni salju samo finalni rezultat Savezu o tome da li je Igrac prosao pregled ili nije zbog poverljivosti podataka.
        \item Sluzbenik - Osoba izabrana od strane Teniskog Saveza Srbije, tj Takmicarskog Odbora u okviru Saveza za prijem uzivo Neregistrovanih clanova, Igraca i Klubova za sve procese koje zele da obave u okviru Saveza i to delegiraju Takmicarskom odboru i popunjavaju rezultate tih procesa u okviru sistema. 
    \end{enumerate}
    
\section{Analiza sistema}
Proucavanjem Teniskog Saveza Srbije je ustanovljeno da je njihov trenutni sistem zastareo i nedovoljno automatizovan. I dalje se vecina dokumenata cuva u pismenoj formi i skenira pa kaci na veb stranicu. Igraci se registruju kao clanovi Saveza tako sto idu i sami se prijavljuju uzivo ili neko umesto njih ide u Savez da ih prijavi, ili se prijave klubu a klub radi pesacki posao. Ovaj nacin rada prouzrokuje nepotrebno trosenje vremena svih aktera u okviru Saveza. Baza podataka deluje funkcionalno samo za pregled rang listi, sto bi trebalo prosiriti na sve aspekte cuvanja podataka o Igracima, Klubovima i rezultatima kroz godine, kao i svom pomocnom osoblju koje ucestvuje u fazama organizacije turnira.\par
Ovaj informacioni sistem omogucava lakse prijavljivanje novih clanova u okviru Saveza, lakse organizovanje Medicinskih pregleda, laksu prijavu za organizaciju turnira i adekvatniji izbor kluba u skladu sa tipom turnira koji treba da se odigra, laksu prijavu na turnire i uvid u njihove detalje organizovanja. Takodje, veci uvid Igraca u njihove rezultate radi moguceg poboljsanja taktike pripreme za turnir ili neki pojedini mec. Savez bi u svakom trenutku imao uvid u sve pojedinosti organizovanja svakog turnira i ko je sve ucestvovao u svakom procesu realizacije. Takodje, sve izmene podataka bi bile lakse i manje papirologije bi bilo u opticaju.Jasno bi se znala uloga svakog clana, efikasnije bi se vrsile dodele pomocnog osoblja jer bi se znalo ko je najadekvatniji da se dodeli organizatoru. \par
U okviru sistema se ne vodi evidencija nicega sto ne interesuje direktno Teniski Savez Srbije, tj Takmicarski odbor. 



\end{document}
